\documentclass[a4 paper , 12pt]{article}
\usepackage[utf8]{inputenc}
\usepackage{xecjk}
\usepackage{pdfpages}
\setCJKmainfont{kaiu.ttf}
\usepackage{setspace}
\onehalfspace
\usepackage{tikz}
\usepackage{graphicx}
\usepackage{moresize}
\usepackage{indentfirst}
\setmainfont{Times New Roman}

\begin{document}

\begin{titlepage}
    \centering
    \includegraphics[width=10CM]{OFTEN/National_Yunlin_University_of_Science_and_Technology_logo.svg.png}\\[1cm]
    \includegraphics[scale=1.5]{OFTEN/yulin.png}
    \vfill
    \LARGE 科系:資訊管理系\\[0.5CM]
    \LARGE 姓名:劉彥辰
\end{titlepage}
\pagenumbering{roman}
\renewcommand{\contentsname}{目錄}

\begin{center}
    \tableofcontents
\end{center}
\pagebreak

\indent\setlength{\parindent}{1em}

\pagenumbering{arabic} 
\section{簡歷}

\begin{figure}[h]

    \centering
    \includepdf[pages=-,pagecommand={},width=21cm,offset=0 -110,]{PDF/簡歷.pdf}
    \label{fig:my_label}
\end{figure}

\pagebreak    

\section{自傳}

\subsection{自我介紹}

\subsubsection{人格特質}
本人具有適應力強,容易溝通,在面對壓力時,仍然可以保持冷靜,快速知道彼此所需,以及描述及提出問題的能力,讓團隊快速了解癥結點後,集思廣益,破除困難,同時我也喜歡當一個風趣好相處的人,我希望能夠與許許多多的人成為朋友。
\subsubsection{樂於溝通}
我喜愛和平,並且非常注重溝通,我認為人們的衝突普遍都發生於溝通上,因此在面對衝突時能夠冷靜聆聽對方的想法,不隨著對方的情緒,而讓溝通只剩下情緒發言,因此能夠達成溝通的最大效益。
\subsubsection{表達能力}
因五專的作業需求,常常需要到人群前演講,所以面對演講時,能夠快速冷靜,盡可能避免把緊張情緒帶給觀眾,並且梳理邏輯,慢慢的表達。

\subsection{就讀動機}
\subsubsection{課程}
之所以想要讀雲林科技大學的資訊管理科,最主要的原因是因為,課程符合我所期望學習的東西,不管是資料庫,網路,甚至是物件導向的概念,都可以從學校獲取資源,也有老師的支援,因此在學習上可以避免自學時的鑽牛角尖,能夠提升學習的效率與興趣。
\subsubsection{同儕}
能夠進入雲科資管的學生,我相信資質一定備受肯定,能夠跟優秀的同學一起進步,一起合作做專題,一定能夠大大提升實力,這些我相信一定比自己單打獨鬥來的更加的有用,也促成我想要進入雲科的原因之一。
\subsubsection{資管系連結}
從五專期間我就略有耳聞雲科資管的訓練扎實,也被叮囑若你覺得你有本事能夠上更好的學校,就應該要去爭取,因此雲林科技大學一直以來都是我的首位學校,在五專所學的東西實在是非常廣闊,卻鮮少有機會能夠深掘某個技術或領域,因此能夠接受完整系統的訓練,一直以來就是我五年來的心願,因此我做了許許多多的努力,不管是考證照,或是學習課外的技能,都是為了可以與更優秀的人一起進步。

\subsection{專題}
\subsubsection{專題貢獻}
我在專題的主要位置是屬於開發遊戲程式,訓練機器學習模型,以及報告,在過程中逐漸發現自己對於程式設計並不排斥,訓練模型時與同學一同探討如何做得更準確,都讓我感到自己有所進步,因此想要繼續往這方面精進。

在專題的途中偶爾會發生隊員之間因意見不合而產生的摩擦,我可以快速統整意見並且想出折衷辦法,讓專題能夠順利如期執行。

\subsubsection{專題成果}
「肢體語言辨識系統之加值應用」獲得資訊智慧創新跨域競賽優等的好成績,以及樹人資管學年度專題競賽的第二名,本次專題除了打下對於程式的邏輯概念,更是讓我學會團隊合作的力量,以及資料蒐集,描述問題的能力。



\subsection{校外經驗}
\subsubsection{實習經驗}
在四年級暑假時有去旭城資訊實習,除了學習了一些網路的概念,以及跟隨內部人員一同前往醫院和學校進行實務外,也學習到了許多溝通的技巧,讓實務能夠執行的快又準確。


\subsubsection{職場經驗}
本人於全聯曾任職9個月,初入職場學習到很多與人的溝通技巧,挫折應對,以及臨場反應,對我來說都是非常難能可貴的經驗,培養負責且穩重的態度是我認為從職場裡獲取最寶貴的經驗。


\subsection{資訊相關技能}

\subsubsection{\LaTeX}
\LaTeX 因它獨特的標記式語言,成為了功能強大的文書排版及論文撰寫軟體,面對要求格式的文件都可以用最快速的速度更改,結合VS Code的套件,已成為本人愛用的文書編輯軟體之一。

\subsubsection{HTML}
本人能夠使用基礎的 HTML 語法,並能夠以些許 CSS 加以裝飾,目前希望能夠繼續學習 JavaScript 以及 RWD 的相關知識,來讓我的網頁變得更加完善。

\subsubsection{Python}
因學校有相關課程因此學習了許多關於 Python 的相關套件,在老師的帶領下使用 Selenium 進行簡易爬蟲以及 Pandas 進行資料分析。
\subsubsection{Java}
因本人想學習有關於物件導向的概念,得知 TQC+ 證照有專門的考試,可以進行認證,也讓我有了動力進行學習,最終如願考取了基礎級的證照。

\subsubsection{文書處理及資料分析}
這個部分的訓練是來自乙級技能檢定的軟體應用,從資料庫的調用,到樞紐分析表的分析,以及最後文書處理的過程,不只培養了對於數字以及題目公式的細心度外,更是加強了對於排版的美感。

\subsubsection{github版本控管}
因專題時期對此沒有過多的了解,因此對於版本控管的概念相當薄弱,在經歷了無數次的檔案損毀,項目遺失,人為失誤,需回復之前版本,等等的麻煩事後,毅然決然在專題結束後去學習這個小巧方便的版本控管軟體,並且透過內建的 gitPage 架設了一個簡易網站 (https://badroomstringman.github.io/myWeb/) 。

\subsubsection{ubuntu作業系統}
對於 Linux 的 ubuntu 作業系統有基礎概念及操作,曾在老師的帶領下使用了 LAMP 架設了資料庫。

\subsection{其他技能}

\subsubsection{Premiere}
自媒體當道的時代,必須學習的一技之長,期望可以透過網路經營自媒體,打造個人品牌,或是應徵剪輯師,為自己的未來開闢不同的道路。
\subsubsection{Photoshop}
因為想要經營網路店舖因此學習此技能,除了可以與自媒體相輔相成外,也可以轉攝影人像修正,將來不管任何行業或多或少都會用到的技能。

\subsubsection{醫療資訊}
在學校的資源輔助下學習了有關於醫療相關的知識,以及法規,並且順利考取醫療資訊管理師證照。

\subsubsection{會計概念}
在學期間,學習了有關於會計的相關知識,並且透過證照考試證明,有獨自完成整個會計循環的能力。


\section{讀書計畫}
% 就讀專科的五年之中,凡事總抱著十足的新鮮感、好奇心,願意嘗試不一樣的課程學習。而若能進入貴系,在將來的學習規劃上我主要分成下列三個階段:
% \subsection{短期}
% 我會先了解系上各方面課程的安排及校園環境,盼能在最短時間內,幫助自己進入狀況。再根據貴系課程上的安排,加強在五專中所學習過的相關課程。\\
% 根據自學一些程式的經驗,首要加強的是自己的英文聽說讀寫的能力,不單減少在學習上及未來所需面對的原文文件的障礙,英文也是日後職場必備的重要能力。

% \subsection{中期}
% 培養扎實的專業基礎能力,並嘗試將課堂上所學到的知識學以致用。充實應具備的專業知識及技能,並充分運用校內的資源。並且在不影響課業的原則下,主動積極參與校內各種活動及社團,拓展自己的人際關係,並學習團體生活中的人際互動,擴大人際關係及生活圈。
% \subsection{長期}
% 結合自身所學,繼續向前邁進,但會根據當時狀況選擇是否繼續升學或從事相關行業,在專業領域上多方面充實內在,逐步建立自己競爭的優勢,持續拓展自我的知識。
\subsection{第一學期}

    \subsubsection{職場英文}
        
        我期望從本課程中學習更多英文,在課堂之餘也希望透過該節課程所學的內容,及善用學校資源,去報考多益。

    \subsubsection{資料庫管理系統}
        五專就學期間,對資料庫有初步的概念,除了嘗試使用 LAMP 建置外,也有進行簡單的新增修改查詢,因此想要繼續更精進,以能夠透過資料庫完成實務為主要目標。

    \subsubsection{資訊網路}
        在4年級時曾經去旭城資訊實習,在那裡我得知思科證照的存在,因此希望能夠在本課程的教學之下,補足網路的知識,以能夠考取CCNA為目標努力。
    
\subsection{第二學期}


    \subsubsection{管理資訊系統}
        
    \subsubsection{資訊管理實務專題}
        五專時,曾與同學一同完成學年度專題,因此對於整體的資料蒐集開會討論等等的流程並不會陌生。   
    \subsubsection{物件導向程式語言}
        我曾考取過相關證照,但是我覺得還是有所不足,自己學習這方面的知識難免會有許多盲點,因此還是需要有相關課程才能夠全方位的補全最正統的物件導向概念。

\subsection{第三學期}


    \subsubsection{企業倫理}
        
    \subsubsection{資訊管理實務專題(二)}
        這學期應該會評估目前專題的進度,期望能夠在雛形階段先獲得一些獎項,若進度可行則去報名較為大型或需要成品的比賽。
    \subsubsection{專案管理}

    \subsubsection{物件導向軟體工程}



\section{證明文件}
\subsection{證照}
\subsubsection{勞動部證照}
\includegraphics[width=13CM]{pro/pp.jpg}
\subsubsection{TQC類}
\includegraphics[width=13cm,height=10CM]{pro/JAVA.jpg}\\
\includegraphics[width=14cm,height=10CM]{pro/IFRS.jpg}
\subsubsection{其他類}
\includegraphics[width=10cm,angle=-90,]{pro/MIM.jpg}

\subsection{獎狀}
\subsubsection{校外}
\includegraphics[width=13cm]{pro/out.jpg}
\subsubsection{校內}
\includegraphics[width=13cm]{pro/in.png}
\centering
\subsection{歷年成績單}
\vfill
\Huge 請翻頁閱讀



\includepdf[pages=-,width=21cm]{PDF/掃描檔864.pdf}


\pagebreak

\begin{center}

% \begin{textblock}{4}(3.8,4)
%%%% 14

\section{其他資料}

\subsection{實習報告書}

% \end{textblock}

\vfill
\Huge 請翻頁閱讀


\includepdf[pages=-]{PDf/實習成果報告書.pdf}



% \begin{textblock}{5}(3,4)
%%%%% 25
\subsection{專題成果報告書}

% \end{textblock}

\vfill
\Huge 請翻頁閱讀

%%%% 62
\includepdf[pages=-]{PDF/成果報告書.pdf}

% \begin{textblock}{4}(3.8,4)



% \end{textblock}

     
\end{center}

\section*{結語} 
\normalsize 孟母三遷的故事告訴我們環境的影響之大,絕對不容小覷,古人有云:近朱者赤;近墨者黑,意指與什麼人相處,自己也會逐漸被同化,若是自己身處於大染缸之中,我並沒有出淤泥而不染的把握與勇氣,為了克服這一點,因此我選擇入靈芝之室,至於何處為靈芝之室呢? 我想我在準備這份書審資料時,已經在心中悄悄有了答案了吧。謝謝看到這裡的老師。




\end{document}